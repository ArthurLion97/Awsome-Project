We attributed water scarcity to uneven distribution in space and time and imbalance between supply and demand. We solved the former by transferring water across regions and storing water for future use, and for the latter we considered supply augmentation and demand constraint methods.
We first used a grey model to predict gap between demand and supply in 2015. Results are that there will be 15 provinces in short of water. Jiangsu province, the most severe case, will be faced with 58.32 billion $m^3$ of water shortage.
Next we developed four models to address water transfer, water storage, desalinization and water conservation. A transportation model was applied to determine an optimal transfer strategy. Results suggested that we transport 12.32 billion $m^3$ of water from the Songliao Region to the Haihe Region, and 5.2 billion $m^3$ of water from the Long River to the Yellow River. We applied a news-vendor model to determine optimal amount of water needed. A case study of Three Gorges Reservoir revealed that 84.1 billion $m^3$ of water should be stored now to satisfy water demand in 2025. A NPV analysis of desalinization projects indicated that 4 desalinization plants should be built in Shanghai, and several more in other provinces in need of water. A Ramsey pricing model was used to determine an optimal pricing strategy. A case study of Shaanxi province revealed that increasing block tariffs achieves a demand reduction of 17.8 $m^3$ per person per year.
Finally, we provided a guide for government to make decisions and propose specific measures for four representative regions. 
Our models are conceptual ones and solutions are based on mathematical optimization. So with more precise data we are able to modify our results without much burdensome repetitions.

