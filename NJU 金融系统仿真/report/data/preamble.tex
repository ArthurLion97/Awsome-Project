%% 下一行说明请见: http://goo.gl/vqz7ps
%\usepackage{fontspec,xunicode,xltxtra}

%% 下一行说明请见:http://goo.gl/tVfJGd
%\usepackage{xeCJK}

%% 字体设置
\setCJKfamilyfont{song}{SimSun}
\setCJKfamilyfont{zhongsong}{STZhongsong}
%\setCJKfamilyfont{kai}{KaiTi}
\setCJKfamilyfont{kai}{KaiTi_GB2312}
\setCJKfamilyfont{hei}{SimHei}
\setCJKfamilyfont{fs}{FangSong_GB2312}
\setCJKfamilyfont{li}{LiSu}
\setCJKfamilyfont{yy}{YouYuan}
\newcommand{\song}{\CJKfamily{song}}
\newcommand{\zs}{\CJKfamily{zhongsong}}
\newcommand{\kai}{\CJKfamily{kai}}
\newcommand{\hei}{\CJKfamily{hei}}
\newcommand{\fs}{\CJKfamily{fs}}
\newcommand{\li}{\CJKfamily{li}}
\newcommand{\yy}{\CJKfamily{yy}}
\setmainfont{Times New Roman} %英文字体使用Times New Roman
%\setmainfont[SmallCapsFont=LMRomanCaps10]{Times New Roman}
%Times New Roman 字体不包括 Small Caps 形状,如果要使用,需设定 Small Caps 字体,如 LMRomanCaps10

\renewcommand{\ULthickness}{0.7pt}
\newcommand{\myuline}[2]      {\uline{\makebox[#1]{#2}}}
\newcommand{\NJUTunderline}[1]{\uline{\hfill{#1}\hfill}}
\footnotesep=10pt

\usepackage{amsmath,amsfonts,amssymb}
\usepackage{array}
\usepackage{booktabs,multirow,colortbl,longtable}
\usepackage{verbatim}
\usepackage{lipsum}
\usepackage{comment}
\usepackage{footnpag}
%\usepackage{mathrsfs}
\usepackage{multirow}

%% 加载图形宏包
\usepackage{graphicx}
%% 设置图片目录
\graphicspath{{figures/}}
\usepackage[config]{subfig}
\usepackage{indentfirst}
%% 紧凑的列表环境
\usepackage[neverdecrease]{paralist}
\let\itemize\compactitem
\let\enditemize\endcompactitem
\let\enumerate\compactenum
\let\endenumerate\endcompactenum
\let\description\compactdesc
\let\enddescription\endcompactdesc

\usepackage[margin=10pt, font=small, labelfont=bf, labelsep=quad]{caption}
%%设置浮动体(表格、图片)标题格式
%\DeclareCaptionLabelFormat{nju}{{\zihao{5}\song #1~#2}}
%\DeclareCaptionLabelSeparator{nju}{\hspace{1em}}
%\DeclareCaptionFont{nju}{\zihao{5}\song}
%\captionsetup{labelformat=nju,labelsep=nju,font=nju}
%\captionsetup[table]{position=top,belowskip={12bp-\intextsep},aboveskip=6bp}
%\captionsetup[figure]{position=bottom,belowskip={12bp-\intextsep},aboveskip=6bp}

%% 超链接、目录
\usepackage{hyperref}
\usepackage{xcolor}
\definecolor{darkblue}{rgb}{0,0,0.55}
\hypersetup{CJKbookmarks,bookmarksnumbered,%
			colorlinks,unicode=true,%
			linkcolor=black,%
			citecolor=darkblue,%
			plainpages=false,%
			bookmarksopen=true,%
			bookmarksopenlevel=1,
			pdfstartview=FitH,
			pdftitle={\thesistitle},
			pdfauthor={\myname},
			pdfcreator={XeLaTeX with NJUThesis template designed by pkuphy},}
\usepackage{tabularx}%只要把tabularx包的引用放到hyperref包之后,正文脚注编号就能正常生成超链接。
%%版面控制
\usepackage{geometry}
\geometry{top=3.5cm,bottom=3.5cm,left=3.2cm,right=3.2cm}
%\geometry{headheight=2.6cm,headsep=5mm,footskip=13mm}
\parskip 0.5ex plus 0.25ex minus 0.25ex

\renewcommand{\textfraction}{0.15}
\renewcommand{\topfraction}{0.85}
\renewcommand{\bottomfraction}{0.65}
\renewcommand{\floatpagefraction}{0.60}


\fancypagestyle{myfancy}{%
\fancyhf{}
\fancyhead[C]{\small \song\leftmark}
\fancyfoot[C]{\small \thepage}
\renewcommand{\headrulewidth}{0.7pt}
}
\pagestyle{myfancy}
\setlength{\headheight}{13.6pt}


\setcounter{secnumdepth}{3}%%自动编号到 subsubsection

\CTEXsetup[nameformat={\hei\zihao{-2}}]{chapter}
\CTEXsetup[titleformat={\hei\zihao{-2}}]{chapter}
\CTEXsetup[beforeskip={-20pt}]{chapter}
\CTEXsetup[afterskip={20pt}]{chapter}
\CTEXsetup[format={\hei\zihao{-3}}]{section}
\CTEXsetup[nameformat={\bf\hei\zihao{-3}}]{section}
\CTEXsetup[beforeskip={-3ex plus -1ex minus -.2ex}]{section}
\CTEXsetup[afterskip={1.0ex plus .2ex}]{section}
\CTEXsetup[format={\hei\zihao{-4}}]{subsection}
\CTEXsetup[nameformat={\bf\hei\zihao{-4}}]{subsection}
\CTEXsetup[beforeskip={-2.5ex plus -1ex minus -.2ex}]{subsection}
\CTEXsetup[afterskip={1.0ex plus .2ex}]{subsection}
\CTEXoptions[contentsname={目\qquad 录}]
\CTEXoptions[listfigurename={插\qquad 图}]
\CTEXoptions[listtablename={表\qquad 格}]

%% 中文破折号,来自清华模板
\newcommand{\pozhehao}{\kern0.3ex\rule[0.8ex]{2em}{0.1ex}\kern0.3ex}
